\documentclass[12pt,a4paper]{article}
\usepackage[utf-8]{inputenc}
\usepackage[russian]{babel}
\usepackage{amsmath}
\usepackage{amssymb}
\usepackage{geometry}
\usepackage{graphicx}
\usepackage{xcolor}
\usepackage{listings}
\usepackage{fancyhdr}
\usepackage{tcolorbox}
\usepackage{tikz}
\usepackage{hyperref}

\geometry{left=2cm, right=2cm, top=2.5cm, bottom=2.5cm}
\setlength{\parindent}{0pt}
\setlength{\parskip}{0.8em}

\lstset{
  language=JavaScript,
  basicstyle=\ttfamily\small,
  breaklines=true,
  keywordstyle=\color{blue},
  commentstyle=\color{gray},
  stringstyle=\color{red},
  numbers=left,
  numberstyle=\tiny\color{gray},
  backgroundcolor=\color{gray!10}
}

\title{\textbf{Компьютерная графика: Параметрические поверхности}}
\author{Полный отчёт о всех заданиях\\Билинейные, Безье, B-сплайновые поверхности}
\date{\today}

\pagestyle{fancy}
\fancyhf{}
\rhead{Компьютерная графика}
\lhead{Параметрические поверхности}
\cfoot{\thepage}

\begin{document}

\maketitle
\tableofcontents
\newpage

% ============================================================================
% ЗАДАНИЕ 1: БИЛИНЕЙНАЯ ПОВЕРХНОСТЬ
% ============================================================================
\section{Задание 1: Построение билинейной поверхности по 4 угловым точкам}

\subsection{Теоретическое введение}

\subsubsection{Определение (академический уровень)}

Билинейная поверхность (bilinear surface) — это параметрическая поверхность степени (1,1), заданная четырьмя угловыми точками $P_{00}, P_{10}, P_{01}, P_{11}$ в трёхмерном пространстве. Она определяется как линейная интерполяция между двумя парами противоположных рёбер.

Поверхность описывается формулой:
\[
S(u,v) = (1-u)(1-v)P_{00} + u(1-v)P_{10} + (1-u)vP_{01} + uvP_{11},
\]
где $u, v \in [0,1]$ — параметры поверхности, а коэффициенты
\[
\begin{aligned}
B_{00}(u,v) &= (1-u)(1-v) \\
B_{10}(u,v) &= u(1-v) \\
B_{01}(u,v) &= (1-u)v \\
B_{11}(u,v) &= uv
\end{aligned}
\]
называются \textit{билинейными базисными функциями}.

\textbf{Свойства:}
\begin{itemize}
  \item Линейна по каждому параметру отдельно.
  \item Образует четырёхугольную панель (patch).
  \item Всегда гладкая $C^0$ (непрерывна по положению), но не гладкая по производным.
  \item Частный случай линейчатой поверхности (ruled surface).
\end{itemize}

\subsubsection{Простое объяснение}

Представьте четыре угловые точки прямоугольника в 3D пространстве. Билинейная поверхность — это поверхность, которая "натягивается" между этими четырьмя углами так, чтобы:
\begin{itemize}
  \item Края поверхности были прямыми линиями (рёбра четырёхугольника).
  \item Поверхность была гладкой "как целлофан".
  \item Каждая точка на поверхности получалась путём "смешивания" четырёх углов.
\end{itemize}

Если сдвинуть один угол вверх, вся поверхность искривится в ту же сторону — это как натянутая ткань на четырёх столбиках.

\subsection{Математическое обоснование}

\subsubsection{Расширенная формула}

Формулу можно переписать в матричной форме:
\[
S(u,v) = \begin{bmatrix} 1-u & u \end{bmatrix} \begin{bmatrix} P_{00} & P_{01} \\ P_{10} & P_{11} \end{bmatrix} \begin{bmatrix} 1-v \\ v \end{bmatrix}
\]

\textbf{Граничные условия:}
\begin{align*}
S(0,0) &= P_{00} \\
S(1,0) &= P_{10} \\
S(0,1) &= P_{01} \\
S(1,1) &= P_{11}
\end{align*}

\textbf{Касательные вектора:}
\[
\frac{\partial S}{\partial u}(u,v) = (1-v)(P_{10}-P_{00}) + v(P_{11}-P_{01})
\]
\[
\frac{\partial S}{\partial v}(u,v) = (1-u)(P_{01}-P_{00}) + u(P_{11}-P_{10})
\]

\subsection{Алгоритм реализации}

\subsubsection{Псевдокод}

\begin{lstlisting}
function bilinearSurface(P00, P10, P01, P11, u, v) {
  w00 = (1 - u) * (1 - v)
  w10 = u * (1 - v)
  w01 = (1 - u) * v
  w11 = u * v
  
  point = w00*P00 + w10*P10 + w01*P01 + w11*P11
  return point
}

function renderBilinear(P00, P10, P01, P11, nu, nv) {
  for u from 0 to 1 step 1/nu:
    line = empty
    for v from 0 to 1 step 1/nv:
      point = bilinearSurface(P00, P10, P01, P11, u, v)
      line.append(point)
    drawLine(line)
  
  for v from 0 to 1 step 1/nv:
    line = empty
    for u from 0 to 1 step 1/nu:
      point = bilinearSurface(P00, P10, P01, P11, u, v)
      line.append(point)
    drawLine(line)
}
\end{lstlisting}

\subsection{Практическое применение}

\begin{tcolorbox}[title=Примеры использования, colback=blue!5]
\begin{itemize}
  \item \textbf{CAD-системы:} Простые панели корпусов, крыши автомобилей.
  \item \textbf{Видеоигры:} Ландшафты с низким разрешением.
  \item \textbf{Архитектура:} Четырёхугольные перекрытия, балконы.
  \item \textbf{Медицина:} Интерполяция данных МРТ между четырьмя точками.
\end{itemize}
\end{tcolorbox}

\subsection{Компьютерная реализация в JavaScript}

Основной код для вычисления точки на поверхности:

\begin{lstlisting}
function evalBilinear(P0, P1, P2, P3, u, v) {
  const w00 = (1-u)*(1-v);
  const w10 = u*(1-v);
  const w01 = (1-u)*v;
  const w11 = u*v;
  
  return {
    x: w00*P0.x + w10*P1.x + w01*P2.x + w11*P3.x,
    y: w00*P0.y + w10*P1.y + w01*P2.y + w11*P3.y,
    z: w00*P0.z + w10*P1.z + w01*P2.z + w11*P3.z
  };
}
\end{lstlisting}

% ============================================================================
% ЗАДАНИЕ 2: ПОВЕРХНОСТЬ ПО ДВУМ ГРАНИЧНЫМ ЛИНИЯМ
% ============================================================================
\newpage
\section{Задание 2: Поверхность между двумя граничными кривыми}

\subsection{Теоретическое введение}

\subsubsection{Определение (академический уровень)}

Линейчатая поверхность (ruled surface) между двумя граничными кривыми $C_0(u)$ и $C_1(u)$ задаётся линейной интерполяцией вдоль параметра $v$:
\[
S(u,v) = (1-v)C_0(u) + vC_1(u), \quad u,v \in [0,1].
\]

Граничные кривые могут быть трёх типов:
\begin{enumerate}
  \item \textbf{Отрезок (линия):} $C(u) = (1-u)P_0 + uP_1$
  \item \textbf{Парабола (квадратичная Безье):} $C(u) = (1-u)^2P_0 + 2(1-u)uP_1 + u^2P_2$
  \item \textbf{Кубическая Безье:} $C(u) = (1-u)^3P_0 + 3(1-u)^2uP_1 + 3(1-u)u^2P_2 + u^3P_3$
\end{enumerate}

\subsubsection{Простое объяснение}

Представьте две кривые в пространстве: верхнюю и нижнюю. Линейчатая поверхность — это поверхность, которая получается, если "натянуть целлофан" между этими двумя кривыми. Каждая точка на поверхности получается простой интерполяцией между соответствующими точками на верхней и нижней кривых.

\subsection{Типы граничных кривых}

\subsubsection{1. Отрезок прямой}

\textbf{Формула:}
\[
C(u) = (1-u)P_0 + uP_1, \quad u \in [0,1]
\]

\textbf{Использует:} 2 контрольные точки.

\textbf{Вид:} Прямая линия.

\subsubsection{2. Парабола (квадратичная кривая Безье)}

\textbf{Формула:}
\[
C(u) = (1-u)^2P_0 + 2(1-u)uP_1 + u^2P_2
\]

где коэффициенты $(1-u)^2, 2(1-u)u, u^2$ — \textit{квадратичные полиномы Бернштейна}.

\textbf{Использует:} 3 контрольные точки ($P_0, P_1, P_2$).

\textbf{Вид:} Гладкая парабола; проходит через $P_0$ и $P_2$, а $P_1$ управляет кривизной.

\subsubsection{3. Кубическая кривая Безье}

\textbf{Формула:}
\[
C(u) = (1-u)^3P_0 + 3(1-u)^2uP_1 + 3(1-u)u^2P_2 + u^3P_3
\]

где коэффициенты — \textit{кубические полиномы Бернштейна} $\binom{3}{k}(1-u)^{3-k}u^k$.

\textbf{Использует:} 4 контрольные точки.

\textbf{Вид:} Гладкая кубическая кривая; очень пластичная, позволяет моделировать S-образные и волнообразные кривые.

\subsection{Полиномы Бернштейна}

\textbf{Общее определение:}
\[
B_{i,n}(t) = \binom{n}{i}(1-t)^{n-i}t^i, \quad i=0,\ldots,n
\]

\textbf{Свойства:}
\begin{itemize}
  \item $\sum_{i=0}^{n} B_{i,n}(t) = 1$ (разбиение единицы).
  \item $B_{i,n}(t) \geq 0$ для $t \in [0,1]$.
  \item $B_{i,n}(0) = \delta_{i,0}$, $B_{i,n}(1) = \delta_{i,n}$ (символ Кронекера).
\end{itemize}

\subsection{Алгоритм построения поверхности}

\begin{lstlisting}
function curvePoint(type, P0, P1, P2, P3, u) {
  if (type == "line")
    return (1-u)*P0 + u*P1
  else if (type == "parabola")
    return (1-u)^2*P0 + 2*(1-u)*u*P1 + u^2*P2
  else if (type == "cubic_bezier")
    return (1-u)^3*P0 + 3*(1-u)^2*u*P1 
         + 3*(1-u)*u^2*P2 + u^3*P3
}

function ruledSurface(C0_type, C1_type, P0..P7, u, v) {
  C0u = curvePoint(C0_type, P0, P1, P2, P3, u)
  C1u = curvePoint(C1_type, P4, P5, P6, P7, u)
  
  S(u,v) = (1-v)*C0u + v*C1u
  return S(u,v)
}
\end{lstlisting}

\subsection{Визуализация 11 вариантов}

Комбинируя три типа кривых для верхней и нижней границы:

\begin{tcolorbox}[title=11 комбинаций поверхностей]
\begin{enumerate}
  \item Отрезок + Отрезок (простейший случай)
  \item Отрезок + Парабола
  \item Отрезок + Кубическая Безье
  \item Парабола + Отрезок
  \item Парабола + Парабола
  \item Парабола + Кубическая Безье
  \item Кубическая Безье + Отрезок
  \item Кубическая Безье + Парабола
  \item Кубическая Безье + Кубическая Безье
  \item S-форма (две разные кубические Безье)
  \item Волна (парабола + кубическая Безье с особыми коэффициентами)
\end{enumerate}
\end{tcolorbox}

% ============================================================================
% ЗАДАНИЕ 3: ПОВЕРХНОСТЬ БЕЗЬЕ
% ============================================================================
\newpage
\section{Задание 3: Бикубическая поверхность Безье}

\subsection{Теоретическое введение}

\subsubsection{Определение (академический уровень)}

Поверхность Безье степени $(3,3)$ (бикубическая) — это параметрическая поверхность, задаваемая сеткой $4 \times 4$ контрольных точек $P_{ij}$ ($i,j = 0,1,2,3$). Поверхность определяется как:
\[
S(u,v) = \sum_{i=0}^{3}\sum_{j=0}^{3} B_{i,3}(u) B_{j,3}(v) P_{ij},
\]
где $B_{i,3}(t)$ — кубические полиномы Бернштейна:
\begin{align*}
B_{0,3}(t) &= (1-t)^3 \\
B_{1,3}(t) &= 3(1-t)^2 t \\
B_{2,3}(t) &= 3(1-t)t^2 \\
B_{3,3}(t) &= t^3
\end{align*}

Контрольные точки образуют "задающий многогранник" (control net), который определяет форму поверхности.

\subsubsection{Простое объяснение}

Поверхность Безье — это трёхмерный аналог гибкой кривой. Вы задаёте 16 точек (4×4 сетка), расположенные в пространстве, а алгоритм строит гладкую поверхность, которая:
\begin{itemize}
  \item Обязательно проходит через 4 угловых точки сетки.
  \item Не обязательно проходит через остальные 12 точек (они управляют кривизной).
  \item При движении одной контрольной точки вся поверхность плавно меняется.
\end{itemize}

Это как натянуть резиновую плёнку на каркас из 16 стержней разной высоты.

\subsection{Математические основы}

\subsubsection{Тензорное произведение}

Поверхность Безье построена методом тензорного произведения двух наборов кривых Безье:
\[
S(u,v) = \left[\sum_{i=0}^{3} B_{i,3}(u) P_{i0}\right] \cdot B_{0,3}(v) + \ldots + \left[\sum_{i=0}^{3} B_{i,3}(u) P_{i3}\right] \cdot B_{3,3}(v)
\]

Это позволяет обрабатывать два параметра независимо, но согласованно.

\subsubsection{Граничные кривые}

Четыре граничные кривые поверхности сами являются кривыми Безье:
\begin{align*}
S(u,0) &= \sum_{i=0}^{3} B_{i,3}(u) P_{i0} & \text{(нижняя кривая)} \\
S(u,1) &= \sum_{i=0}^{3} B_{i,3}(u) P_{i3} & \text{(верхняя кривая)} \\
S(0,v) &= \sum_{j=0}^{3} B_{j,3}(v) P_{0j} & \text{(левая кривая)} \\
S(1,v) &= \sum_{j=0}^{3} B_{j,3}(v) P_{3j} & \text{(правая кривая)}
\end{align*}

\subsection{Свойства поверхности Безье}

\begin{tcolorbox}[title=Ключевые свойства]
\begin{enumerate}
  \item \textbf{Интерполяция углов:} $S(i,j) = P_{ij}$ для $i,j \in \{0,3\}$.
  \item \textbf{Инвариантность к аффинным преобразованиям:} Если преобразовать все $P_{ij}$, то преобразуется и вся поверхность.
  \item \textbf{Локальность:} Движение одной точки влияет на локальную область.
  \item \textbf{Гладкость:} Класс $C^\infty$ везде внутри области $(u,v) \in (0,1)^2$.
  \item \textbf{Выпуклость:} Поверхность находится в выпуклой оболочке своих контрольных точек.
\end{enumerate}
\end{tcolorbox}

\subsection{Алгоритм вычисления точки}

\begin{lstlisting}
function bezier3(P0, P1, P2, P3, t) {
  s = 1 - t
  b0 = s^3
  b1 = 3*s^2*t
  b2 = 3*s*t^2
  b3 = t^3
  return b0*P0 + b1*P1 + b2*P2 + b3*P3
}

function evalBezierSurface(P[4][4], u, v) {
  // P[i][j] — контрольные точки
  x = y = z = 0
  for i in 0..3:
    bu = bernstein3[i](u)  // B_{i,3}(u)
    for j in 0..3:
      bv = bernstein3[j](v)  // B_{j,3}(v)
      weight = bu * bv
      x += weight * P[i][j].x
      y += weight * P[i][j].y
      z += weight * P[i][j].z
  return {x, y, z}
}
\end{lstlisting}

\subsection{Практический пример}

В коде используются 4 "типовые" сетки контрольных точек:

\begin{tcolorbox}[title=Примеры форм]
\begin{itemize}
  \item \textbf{Плоскость:} Все Z = 0.
  \item \textbf{Купол:} $Z_{ij} = A(1 - r^2)$ где $r^2 = (u_i-0.5)^2 + (v_j-0.5)^2$.
  \item \textbf{Седло:} $Z_{ij} = A(u_i^2 - v_j^2)$.
  \item \textbf{Волна:} $Z_{ij} = A\sin(\pi u_i)\sin(\pi v_j)$.
\end{itemize}
\end{tcolorbox}

% ============================================================================
% ЗАДАНИЕ 4: B-СПЛАЙНОВАЯ ПОВЕРХНОСТЬ
% ============================================================================
\newpage
\section{Задание 4: B-сплайновая поверхность}

\subsection{Теоретическое введение}

\subsubsection{Определение (академический уровень)}

B-сплайновая поверхность (B-spline surface) — это параметрическая поверхность, определяемая сеткой контрольных точек $P_{ij}$ (где $i = 0,\ldots,n$, $j = 0,\ldots,m$) и двумя узловыми векторами $U$ и $V$. Поверхность задаётся как:
\[
S(u,v) = \sum_{i=0}^{n}\sum_{j=0}^{m} N_{i,p}(u) M_{j,q}(v) P_{ij},
\]
где:
\begin{itemize}
  \item $N_{i,p}(u)$ и $M_{j,q}(v)$ — B-сплайн базисные функции степени $p$ и $q$ соответственно.
  \item $U = [u_0, u_1, \ldots, u_{n+p+1}]$ и $V = [v_0, v_1, \ldots, v_{m+q+1}]$ — узловые векторы.
\end{itemize}

\subsubsection{Простое объяснение}

B-сплайновая поверхность — это "улучшенная" версия поверхности Безье. Если Безье привязана к контрольной сетке глобально, то B-сплайновая позволяет:
\begin{itemize}
  \item Менять форму локально (сдвиг одной точки влияет только на соседнюю область).
  \item Использовать разные степени гладкости.
  \item Управлять тем, как точно поверхность проходит через контрольные точки (через узловый вектор).
\end{itemize}

Это как иметь много "мягких панелей" с независимым управлением каждой.

\subsection{B-сплайн базисные функции}

\subsubsection{Рекурсия Кокса–де Бура}

B-сплайн базисные функции вычисляются рекурсивно. Базис степени 0:
\[
N_{i,0}(u) = \begin{cases}
1, & u_i \leq u < u_{i+1} \\
0, & \text{иначе}
\end{cases}
\]

Базис степени $p > 0$ через рекурсию Кокса–де Бура:
\[
N_{i,p}(u) = \frac{u - u_i}{u_{i+p} - u_i} N_{i,p-1}(u) + \frac{u_{i+p+1} - u}{u_{i+p+1} - u_{i+1}} N_{i+1,p-1}(u)
\]

\textbf{Замечание:} При делении на ноль полагаем $\frac{0}{0} = 0$.

\subsubsection{Свойства базисных функций}

\begin{tcolorbox}[title=Свойства B-сплайнов]
\begin{enumerate}
  \item \textbf{Разбиение единицы:} $\sum_{i=0}^{n} N_{i,p}(u) = 1$ для всех $u \in [U_p, U_{n+1}]$.
  \item \textbf{Неотрицательность:} $N_{i,p}(u) \geq 0$.
  \item \textbf{Локальность:} $N_{i,p}(u) = 0$ для $u \notin [u_i, u_{i+p+1})$.
  \item \textbf{Гладкость:} Класс $C^{p-k}$ в узлах кратности $k$.
\end{enumerate}
\end{tcolorbox}

\subsection{Узловые векторы}

\subsubsection{Определение}

Узловой вектор $U = [u_0, u_1, \ldots, u_r]$ — это неубывающая последовательность вещественных чисел, которая определяет "границы влияния" базисных функций. Для $n+1$ контрольных точек и степени $p$:
\[
r = n + p + 1
\]

\subsubsection{Типы узловых векторов}

\begin{enumerate}
  \item \textbf{Равномерный (uniform):}
  \[
  U = [0, 0, \ldots, 0, \frac{1}{m}, \frac{2}{m}, \ldots, \frac{m-1}{m}, 1, 1, \ldots, 1]
  \]
  где ноль и один повторены $p+1$ раз.

  \item \textbf{Зажатый (clamped/open-uniform):}
  \[
  U = [\underbrace{0,\ldots,0}_{p+1}, u_{p+1}, \ldots, u_r-p, \underbrace{1,\ldots,1}_{p+1}]
  \]
  Первый и последний узлы повторяются $p+1$ раз, что "прижимает" поверхность к угловым контрольным точкам.

  \item \textbf{Неравномерный (non-uniform):}
  Произвольные значения для специальных эффектов (редко используется в базовых приложениях).
\end{enumerate}

\subsubsection{Зажатый узловой вектор для 4 контрольных точек, степень 3}

Для $n+1 = 4$ (индексы 0..3) и $p = 3$:
\[
r = 3 + 3 + 1 = 7
\]

Зажатый вектор:
\[
U = [0, 0, 0, 0, 1, 1, 1, 1]
\]

Это гарантирует, что B-сплайновая поверхность совпадает с поверхностью Безье!

\subsection{Практическое сравнение}

\begin{tcolorbox}[title=Безье vs B-сплайн]
\begin{tabular}{|l|l|l|}
\hline
Параметр & Безье & B-сплайн \\
\hline
Интерполяция углов & Да (всегда) & Зависит от узлов \\
Локальность & Ограниченная & Полная \\
Гладкость & Встроенная & Управляемая \\
Степень сложности & Простая & Средняя \\
Практическое применение & Дизайн & CAD-системы \\
\hline
\end{tabular}
\end{tcolorbox}

\subsection{Алгоритм вычисления}

\begin{lstlisting}
function bsplineBasis(i, p, u, U) {
  // Cox-de Boor recursion
  if (p == 0) {
    if (U[i] <= u && u < U[i+1]) return 1
    else return 0
  }
  
  const denom1 = U[i+p] - U[i]
  const denom2 = U[i+p+1] - U[i+1]
  
  const a = (denom1 != 0) 
    ? ((u - U[i]) / denom1) * bsplineBasis(i, p-1, u, U)
    : 0
  const b = (denom2 != 0)
    ? ((U[i+p+1] - u) / denom2) * bsplineBasis(i+1, p-1, u, U)
    : 0
  
  return a + b
}

function evalBSplineSurface(P, u, v, p, q, U, V) {
  let x = 0, y = 0, z = 0
  
  for (let i = 0; i < P.length; i++) {
    const Nu = bsplineBasis(i, p, u, U)
    if (Nu == 0) continue
    
    for (let j = 0; j < P[0].length; j++) {
      const Nv = bsplineBasis(j, q, v, V)
      if (Nv == 0) continue
      
      const weight = Nu * Nv
      const pij = P[i][j]
      x += weight * pij.x
      y += weight * pij.y
      z += weight * pij.z
    }
  }
  
  return {x, y, z}
}
\end{lstlisting}

% ============================================================================
% ТРЁХМЕРНЫЕ ПРЕОБРАЗОВАНИЯ
% ============================================================================
\newpage
\section{Приложение: Трёхмерные преобразования и проецирование}

\subsection{Матрицы вращения}

Вращение вокруг оси $X$ на угол $\theta$ (в радианах):
\[
R_X(\theta) = \begin{bmatrix}
1 & 0 & 0 \\
0 & \cos\theta & -\sin\theta \\
0 & \sin\theta & \cos\theta
\end{bmatrix}
\]

Вращение вокруг оси $Y$ на угол $\phi$:
\[
R_Y(\phi) = \begin{bmatrix}
\cos\phi & 0 & \sin\phi \\
0 & 1 & 0 \\
-\sin\phi & 0 & \cos\phi
\end{bmatrix}
\]

Композиция: $R = R_Y(\phi) \cdot R_X(\theta)$ (порядок важен!).

\subsection{Ортографическое проецирование}

После применения вращения точка $P' = R \cdot P$ проецируется на плоскость экрана:
\[
\begin{aligned}
x_{\text{screen}} &= c_x + s \cdot p'_x \\
y_{\text{screen}} &= c_y - s \cdot p'_y
\end{aligned}
\]

где $c_x, c_y$ — центр экрана, $s$ — масштаб, а $p'_x, p'_y$ — координаты в повёрнутой системе.

% ============================================================================
% ЗАКЛЮЧЕНИЕ
% ============================================================================
\newpage
\section{Заключение и сводка результатов}

\subsection{Обзор всех поверхностей}

\begin{tcolorbox}[title=Сравнение всех типов поверхностей]
\begin{tabular}{|l|l|l|l|}
\hline
Тип & Параметры & Использование & Формула \\
\hline
Билинейная & 4 точки & Простые панели & $(1-u)(1-v)P_{00} + \ldots$ \\
Линейчатая & 2 кривые & Переходы & $(1-v)C_0(u) + vC_1(u)$ \\
Безье (4×4) & 16 точек & Дизайн & $\sum B_i(u)B_j(v)P_{ij}$ \\
B-сплайн & $n \times m$ точек & CAD & $\sum N_i(u)M_j(v)P_{ij}$ \\
\hline
\end{tabular}
\end{tcolorbox}

\subsection{Практические рекомендации}

\begin{itemize}
  \item Для быстрого прототипирования: \textbf{билинейная поверхность}.
  \item Для сложного дизайна: \textbf{поверхность Безье}.
  \item Для производственных систем: \textbf{B-сплайновая поверхность}.
  \item Для переходов между объектами: \textbf{линейчатая поверхность}.
\end{itemize}

\subsection{Дальнейшее развитие}

Естественные расширения:
\begin{enumerate}
  \item NURBS-поверхности (рациональные B-сплайны).
  \item Поверхности Кунса (Coons surfaces) с четырьмя граничными кривыми.
  \item Подразделяемые поверхности (subdivision surfaces).
  \item Неявные поверхности и неявный дизайн.
\end{enumerate}

\end{document}
